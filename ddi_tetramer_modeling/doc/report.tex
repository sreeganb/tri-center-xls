%---------------------------------------------------------------------------
\documentclass[11pt,a4paper]{report}

% Page geometry and layout
\usepackage[margin=1in, headheight=14pt]{geometry}

% Font and typography
\usepackage[T1]{fontenc}
\usepackage{newtxtext,newtxmath}  % Professional Times-like fonts
\usepackage{microtype}            % Better text spacing and justification

% Math and symbols
\usepackage{amsmath}
\usepackage{amsfonts}
\usepackage{mathtools}

% Graphics and figures
\usepackage{graphicx}
\usepackage{float}                % Better figure placement
\usepackage{subcaption}           % Subfigures
\usepackage{tikz}                 % For diagrams

% Tables
\usepackage{booktabs}             % Professional tables
\usepackage{array}
\usepackage{multirow}

% Units and formatting
\usepackage{siunitx}              % Scientific units
\usepackage{physics}              % Physics notation
\usepackage{bm}                   % Bold math

% Section formatting
\usepackage{titlesec}
\titlespacing*{\section}{0pt}{5.5ex plus 1ex minus .2ex}{4.3ex plus .2ex}
\titlespacing*{\subsection}{0pt}{4.5ex plus 1ex minus .2ex}{3.3ex plus .2ex}
\titlespacing*{\subsubsection}{0pt}{3.5ex plus 1ex minus .2ex}{2.3ex plus .2ex}

% Lists
\usepackage{enumitem}
\setlist[itemize]{leftmargin=*, itemsep=0pt, topsep=3pt}
\setlist[enumerate]{leftmargin=*, itemsep=0pt, topsep=3pt}

% Hyperlinks and references
\usepackage[hypertexnames=false]{hyperref}
\hypersetup{
    colorlinks=true,
    linkcolor=blue,
    citecolor=blue,
    urlcolor=blue,
    pdftitle={DDI Protein Complex Structure: Quantitative Crosslinking Mass Spectrometry Analysis},
    pdfauthor={Sree Ganesh Balasubramani},
    pdfsubject={Structural Biology},
    pdfkeywords={protein complex, crosslinking, mass spectrometry}
}

% Code listings (if needed)
\usepackage{listings}
\usepackage{xcolor}
\lstset{
    basicstyle=\ttfamily\footnotesize,
    breaklines=true,
    captionpos=b,
    numbers=left,
    numberstyle=\tiny,
    frame=single,
    backgroundcolor=\color{gray!10}
}

% Headers and footers
\usepackage{fancyhdr}
\pagestyle{fancy}
\fancyhf{}
\fancyhead[L]{\leftmark}
\fancyhead[R]{\thepage}
\renewcommand{\headrulewidth}{0.4pt}

% Title page customization
\usepackage{titling}
\pretitle{\begin{center}\LARGE\bfseries}
\posttitle{\end{center}\vspace{1em}}
\preauthor{\begin{center}\large}
\postauthor{\end{center}}
\predate{\begin{center}\large}
\postdate{\end{center}}

% Abstract environment
\usepackage{abstract}
\renewcommand{\abstractnamefont}{\normalfont\bfseries}
\renewcommand{\abstracttextfont}{\normalfont}

% Final touches
\usepackage{setspace}             % Line spacing
\onehalfspacing                   % 1.5 line spacing

\author{Sree Ganesh Balasubramani}
\title{DDI Protein Complex Structure:\\Quantitative Crosslinking Mass Spectrometry Analysis}

\begin{document}

% Title page
\maketitle

\chapter{Methods}
\label{chap:methods}

\section{Protein Production and Labeling}
\label{sec:labeling}

Two separate batches of the protein are produced under controlled conditions:

\begin{itemize}
    \item \emph{Light batch}: Grown in normal media containing light amino 
    acids such as lysine with $^{12}$C and $^{14}$N isotopes.
    \item \emph{Heavy batch}: Grown in media enriched with heavy 
    isotopes ($^{13}$C and $^{15}$N) for quantitative mass spectrometry analysis.
\end{itemize}
Crucially the light and the heavy proteins are chemically identical, and fold 
the same way, but the heavy version weighs slightly more and the mass spectrometer 
can distinguish between the two.

\section{Mixing and crosslinking}
\label{sec:mixing}
The purified light and heavy proteins are mixed in a 1:1 ratio. This mixture now contains 
a pool of both heavy and light subunits, that assemble into tetramers. 
Then the crosslinker is added.

\section{Possible outcomes}
The mass spectrometer can distinguish three different types of crosslinks based on 
their mass:
\begin{itemize}
\item \emph{Light-Light (L-L)}: Both subunits in the crosslink are light.
\item \emph{Heavy-Heavy (H-H)}: Both subunits in the crosslink are heavy.
\item \emph{Light-Heavy (L-H)}: One subunit is light and the other is heavy.
\end{itemize}
The \emph{L-H} crosslinks are the most informative, as they are the most 
unambiguous for an inter-subunit crosslink. It is impossible to get a mixed pair 
from a crosslink happening within a single protein copy. 
\end{document}