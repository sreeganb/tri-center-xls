\documentclass[a4paper,8pt]{beamer}
%\documentclass{beamer}
%\input{embed_video.tex}
\mode<presentation>
%{
  \usetheme{Boadilla}
  % possiblities Singapore, Malmoe, Dresden 
  \setbeamercovered{invisible}
%}
%color setting more or less matching U of A colors
\setbeamercolor*{palette secondary}{use=structure,fg=white,bg=structure.fg!55!black}
\setbeamercolor*{palette tertiary}{use=structure,fg=white,bg=red!50!black}

% Define a sober color theme
\definecolor{mycolor}{RGB}{0,102,204}
\setbeamercolor{item}{fg=mycolor}
\setbeamercolor{section in toc}{fg=mycolor}
\setbeamercolor{footline}{bg=mycolor, fg=white}
\setbeamercolor{headline}{bg=mycolor, fg=white}
\setbeamercolor{title}{fg=mycolor}
\setbeamercolor{author}{fg=mycolor}

%\usetheme{CambridgeUS}
%\usecolortheme{wolverine}
\usepackage[utf8]{inputenc} % Load inputenc before biblatex
\usepackage[
    backend=biber,
%    style=phys,
%    pageranges=false,
%    biblabel=brackets,
%    chaptertitle=false,
%    articletitle=false,
    maxbibnames=3 %,
%    doi=false, url=false, isbn=false
]{biblatex}
\usepackage{graphicx}
\usepackage{bm}
\usepackage{tabularx,booktabs}
\usepackage{subcaption}
\usepackage{hyperref}
\setbeamertemplate{footline}[frame number]
\usepackage{tikz}
\usetikzlibrary{arrows}
\definecolor{myblue}{RGB}{5,32,73}
\newcommand\xsetpos{6}
\usepackage{epsfig}
\usepackage{amsmath}
\usepackage{xcolor}

\addbibresource{presentation.bib}
\usetikzlibrary{arrows}
\beamertemplatenavigationsymbolsempty
 \definecolor{dgreen}{rgb}{0.,0.6,0.}
\newcolumntype{d}[1]{D{.}{\cdot}{#1}}
%------------------------------------------------------------------------------
\title{Updates}
\author[]{Sree Ganesh Balasubramani}
\institute[UCSF]{Echeverria Group, UC San Francisco}
\date{}
% This is to display the outline at the beginning of each section
\AtBeginSection[]
{
  \begin{frame}{Outline}
    \tableofcontents[currentsection]
  \end{frame}
}
%\renewcommand\appendixname{Appendix}
\usepackage{pgf}
\logo{\pgfputat{\pgfxy(0.15,8)}{\pgfbox[right,base]{\includegraphics[height=1.25cm]{figures/ucsf-logo.eps}}}}
\newcommand{\nologo}{\setbeamertemplate{logo}{}}
%------------------------------------------------------------------------------
\begin{document}
\maketitle
%
\begin{frame}
    \begin{figure}
    \centering
    \includegraphics[width=0.5\textwidth]{figures/5gjr_structure.png}
    \caption{EM structure at 3.5 {\AA} resolution of the human 26S proteasome, PDB ID: 5GJR}
    \end{figure}
    % a table with two rows and two columns
\begin{table}
    \centering
    \caption{Number of unique cross links}
    \begin{tabular}{|c|c|}
        \hline
        Type of XL & Number of unique XLs \\ \hline
        Triple & 35 \\ \hline
        Double & 691 \\ \hline
    \end{tabular}
\end{table}
\end{frame}
%
\begin{frame}
    \frametitle{Working with the base subcomplex}
\begin{figure}
\centering
\includegraphics[width=0.5\textwidth]{figures/base_proteasome_5gjr.png}
\caption{Base subcomplex consisting of Rpt1, Rpt2, Rpt3, Rpt4, Rpt5, Rpt6, Rpn2 of subunits 
of the human 26S proteasome}
  \end{figure}
\end{frame}
%
\begin{frame}
\frametitle{Comparison of the double and triple XL datasets}
\begin{block}{}
\begin{itemize}
  \item Only bifunctional set ($n(bi) = 164$)
  \item Only trifunctional set ($n(tri)= 42$)
  \item bifunctional + trifunctional set ($n(bi) + n(tri) = 206$)
  \item n($bi \cup tri) = n(bi) + n(tri) - n(bi\cap tri) = 186$
\end{itemize}
\end{block}
    \begin{figure}
      \centering
      \includegraphics[width=0.43\textwidth]{figures/only-doubles.png}
      \includegraphics[width=0.43\textwidth]{figures/removed-doubles-plus-triples.png}
      \caption{$n(bi) = 164$ (left) and n($bi \cup tri = 186$) (right) set of cross links mapped onto the base subcomplex}
    \end{figure}
\end{frame}
%
\begin{frame}
\frametitle{Comparison of the double and triple XL datasets}
\begin{table}
  \centering
  \caption{Comparison of the double and triple XL datasets}
  \begin{tabular}{|c|c|c|c|}
      \hline
                                   & Samp. precision ({\AA}) & Clust. pop. & Clust. precision\\ \hline
      bifunctional ($n(bi) = 164$) & 4.42  & 21820 & 3.335 \\ \hline
      $n(bi) + n(tri) = 206$       & 5.58  & 28449 & 3.666 \\ \hline
      n($bi \cup tri) = 186$       & 10.68 & 29395 & 7.388 \\ \hline
  \end{tabular}
\end{table}
Clustering at sampling precision $6.0$ {\AA}
\begin{table}
  \centering
  \caption{Comparison of the double and triple XL datasets}
  \begin{tabular}{|c|c|c|c|}
      \hline
                                   & Clust. pop. & Clust. precision\\ \hline
      bifunctional ($n(bi) = 164$) & 27397 & 3.755 \\ \hline
      $n(bi) + n(tri) = 206$       &  &  \\ \hline
      n($bi \cup tri) = 186$       & 24090 & 4.118 \\ \hline
  \end{tabular}
\end{table}
\begin{block}{}
  Remove the excluded volume for the dummy beads and the distribution of the XL distances 
\end{block}
\end{frame}
%
\begin{frame}
\frametitle{XL residue pair distance distributions}
\begin{figure}
\centering
\includegraphics[width=0.45\textwidth]{figures/Rpt2_258_to_Rpt4_72_distances_distribution.pdf}
\includegraphics[width=0.45\textwidth]{figures/Rpt2_258_to_Rpt5_300_distances_distribution.pdf}
\end{figure}
%
\begin{figure}
  \centering
  \includegraphics[width=0.45\textwidth]{figures/Rpt6_222_to_Rpt2_258_distances_distribution.pdf}
  \includegraphics[width=0.45\textwidth]{figures/Rpt6_222_to_Rpt5_300_distances_distribution.pdf}
  \end{figure}  
\end{frame}
%
\begin{frame}
\frametitle{DSSO crosslinking data}
\begin{figure}
\centering
\includegraphics[width=0.5\textwidth]{figures/dsso-mapped-xls.png}
\caption{DSSO crosslinking data mapped onto the base subcomplex}
\end{figure}
\end{frame}
%\begin{frame}
%\frametitle{PSD modeling}
%Modeling intra-protein cross links for O08749 interactor protein of Bassoon. 
%\begin{figure}
%\centering
%\includegraphics[width=0.5\textwidth]{figures/xls_mapped.png}
%\includegraphics[width=0.5\textwidth]{figures/distances_O08749.pdf}
%\end{figure}
%\end{frame}
%\begin{frame}
%\frametitle{PSD modeling}
%Using Alphafold multimer to obtain the homodimer strucutre and map the XLs onto it. 
%\end{frame}
\end{document}