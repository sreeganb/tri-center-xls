\documentclass[a4paper,8pt]{beamer}
\mode<presentation>
{
  \usetheme{Boadilla}
  \setbeamercovered{invisible}
}

% --- Official UCSF Color Palette ---
\definecolor{ucsfNavy}{RGB}{5, 32, 73}
\definecolor{ucsfBlueGray}{RGB}{80, 99, 128}
\definecolor{ucsfDarkGray}{RGB}{135, 141, 150}
\definecolor{ucsfCoolGray}{RGB}{180, 185, 191}
\definecolor{ucsfGray}{RGB}{209, 211, 211}
\definecolor{ucsfLightGray}{RGB}{225, 227, 230}
\definecolor{ucsfOffWhite}{RGB}{242, 243, 244}

% --- Apply UCSF Colors to Your Working Theme ---
\setbeamercolor*{palette secondary}{use=structure,fg=white,bg=ucsfBlueGray}
\setbeamercolor*{palette tertiary}{use=structure,fg=white,bg=ucsfNavy}
\setbeamercolor*{structure}{fg=ucsfNavy}
\setbeamercolor*{title}{fg=ucsfNavy}

% Block colors with UCSF palette
\setbeamercolor{block title}{bg=ucsfBlueGray, fg=white}
\setbeamercolor{block body}{bg=ucsfOffWhite, fg=black}
\setbeamercolor{block title example}{bg=ucsfCoolGray, fg=white}
\setbeamercolor{block body example}{bg=ucsfCoolGray!20, fg=black}
\setbeamercolor{block title alerted}{bg=ucsfNavy, fg=white}
\setbeamercolor{block body alerted}{bg=ucsfNavy!10, fg=black}

% Required packages
\usepackage[utf8]{inputenc}
% COMMENT OUT BIBLIOGRAPHY FOR NOW - uncomment when you have presentation.bib
%\usepackage[backend=bibtex,maxbibnames=3]{biblatex}
%\addbibresource{presentation.bib}
\usepackage{graphicx,bm,tabularx,booktabs,subcaption,hyperref}
\usepackage{verbatim,adjustbox}
\usepackage[most]{tcolorbox}
\usepackage{epsfig,amsmath,xcolor,listings,caption}
\usepackage{empheq}
\usepackage{pgfplots}
\pgfplotsset{compat=newest}
\usepackage{array} % For newcolumntype
\usepackage{tikz}
\usetikzlibrary{shapes, arrows, positioning}

% Your existing settings that work
\setbeamertemplate{footline}[frame number]
\usepackage{tikz}
\usetikzlibrary{arrows,calc}
\definecolor{myblue}{RGB}{5,32,73}  % This matches ucsfNavy
\newcommand\xsetpos{6}
\setbeamercovered{transparent}
\beamertemplatenavigationsymbolsempty
\definecolor{dgreen}{rgb}{0.,0.6,0.}
\newcolumntype{d}[1]{D{.}{\cdot}{#1}}

% Custom commands with UCSF colors
\newcommand{\highlight}[1]{\colorbox{ucsfGray!30}{#1}}
\newcommand{\important}[1]{{\color{ucsfNavy}\textbf{#1}}}
\newcommand{\itemPause}{\pause\item}

% Highlight boxes
\newtcolorbox{highlightbox}[1][]{
  enhanced,
  colback=ucsfOffWhite,
  colframe=ucsfBlueGray,
  boxrule=0.5pt,
  arc=2mm,
  #1
}

% COMMENT OUT LOGO FOR NOW - uncomment when you have the logo file
%\usepackage{pgf}
%\logo{\pgfputat{\pgfxy(0.15,8)}{\pgfbox[right,base]{\includegraphics[height=1.25cm]{figures/ucsf-logo.pdf}}}}
%\newcommand{\nologo}{\setbeamertemplate{logo}{}}

% Document metadata
\title{Group meeting presentation}
\author[S. G. Balasubramani]{Sree Ganesh Balasubramani}
\institute[UCSF]{Echeverria \& {\v S}ali Groups \\ University of California, San Francisco}
\date{July 26, 2025}

% Section transitions
\AtBeginSection[]
{
  \begin{frame}{Outline}
    \tableofcontents[currentsection]
  \end{frame}
}

\begin{document}

% Title slide
\maketitle

%----------------------------------------------------------------------------
\section{Trifunctional crosslinkers}
%----------------------------------------------------------------------------
\begin{frame}
  \frametitle{Structure of the trifunctional crosslinker}
  
  \begin{figure}
    \centering
    \includegraphics[width=0.4\textwidth]{figures/tri-linker.eps}
    \caption{Structure of TSTO MS cleavable crosslinker with a spacer arm length 
    of 14.0 \AA.}
  \end{figure}
  \begin{block}{}
    For comparison, the commonly used bifunctional crosslinker
  DSSO has a spacer arm length of 10.1 \AA.
    \end{block}
\end{frame}
%----------------------------------------------------------------------------
\begin{frame}
\frametitle{Riccardo's noise model for the crosslinking restraint}
\begin{equation}
p(o_n|X) \propto \frac{\alpha}{\beta}p(XL_n|X) + \frac{1 - \alpha}{1 - \beta}p(\bar{XL_n}|X)
\end{equation}
where 
\begin{equation}
\alpha = \frac{N^{obs, T}_{XL}}{N^{obs}_{XL}} \qquad \beta = \frac{N_{XL}}{N_{LP}}
\end{equation}
There are some constraints on the values of $\alpha$ and $\beta$:   
\begin{equation}
    \frac{\alpha}{\beta} \leq \frac{N_{LP}}{N_{XL}} ;\quad \frac{1-\alpha}{1-\beta} \leq \frac{N_{LP}}{N_{XL}}
\end{equation}

For trifunctional crosslinks, the $\beta$ parameter is modified to be 
\begin{equation}
\beta = \frac{N_{XL}^{tri}}{N_{LT}}
\end{equation}
where the number of lysine triplets $N_{LT}$ is 
given by $^{N_{lysines}}C_3 = \frac{N_{lysines}*(N_{lysines}-1)*(N_{lysines}-2)}{3!}$
\end{frame}
%----------------------------------------------------------------------------
\begin{frame}
\frametitle{Likelihood landscape}
Consider 100 Lysines, 4950 lysine pairs, 100 observed crosslinks
\begin{block}{Assumptions}
\begin{itemize}
\item Consider that all the crosslinks are having the same forward model probability, $f_{mod}$ 
\item Also assume that all the crosslinks have the same value of $\alpha$ and $\beta$
\end{itemize}
\end{block}
When it comes to trifunctional crosslinks, the trifunctional crosslinker can link 3 lysines, 
or if there is a single dead end then it can link 2 lysines and it will be identified as a bifunctional crosslink.
\end{frame}
%======================================================================
\end{document}
%**********************************************************************
